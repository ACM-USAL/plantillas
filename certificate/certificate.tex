% !TEX encoding = UTF-8 Unicode
\documentclass[spanish]{article}

% Ajustes de idioma (añade soporte para la división de palabras al final de linea	y otros ajustes dependientes del idioma)
\usepackage[spanish]{babel}
\selectlanguage{spanish}

% Codificación de caracteres
\usepackage[T1]{fontenc}
\usepackage[utf8]{inputenc}

% Geometría del papel. A4 apaisado
\usepackage[a4paper,top=2.4cm, left=2cm, right=2cm, bottom=2cm, landscape %, showframe %Descomentar para imprimir los márgenes (depuración)
]{geometry}

% Aumenta el espacio disponible al eliminar el espacio reservado para la cabecera
\setlength{\headsep}{0pt}

% Genera Lorem Ipsum (pruebas)
\usepackage{lipsum}

% Inclusión de imágenes
\usepackage{graphicx}
% Rutas de búsqueda de imágenes
\graphicspath{{img/}}

% Fuente arial. Es necesario instalar previamente las fuentes no libres disponibles en CTAN (ver README)
\usepackage[scaled]{uarial}
\renewcommand*\familydefault{\sfdefault}

% Desactiva la enumeración de páginas (por defecto en la clase article)
\pagenumbering{gobble}

\begin{document}

\hbox{
	\includegraphics[height=3.2cm]{usallogo}
	\hspace{14.2cm}
	\raisebox{0.35\height}{\includegraphics[height=2cm]{acmlogo}}
}
\vspace{0.5cm}
\begin{center}

{\LARGE \textbf{CERTIFICADO DE ASISTENCIA}\\[0.5cm]}
{\Large OTORGADO A D./Da. \textbf{Nombre.. Apellidos..}\\[0.2cm]}
{Con D.N.I. DNI..\\[0.3cm]}
{\large Por la participación en el \textbf{DESARROLLO DE VIDEOJUEGOS CON EL MOTOR GRÁFICO UNITY3D}, presentado por el Servicio de Formación Permanente de la Universidad de Salamanca y el Capítulo de Estudiantes ACM USAL, celebrado el día 10 de marzo de 2015 en la Facultad de Ciencias de la Universidad de Salamanca, con una duración de 4 horas, de acuerdo con el programa que figura en el reverso de este documento.\\[0.5cm]
Y para que conste, se expide esta certificación en Salamanca, a 10 de marzo de 2014.\\}

\vspace*{2cm}

% Espacio para las firmas
\parbox{0.9\textwidth}{
	\parbox{7cm}{
		\centering
		Directora del seminario\\
		\vspace*{2cm}
		Dra. María Angélica González Arrieta\\
		Departamento de Informática y Automática\\
		Universidad de Salamanca
	}
	\hfill
	\parbox{7cm}{
		\centering
		El Rector,P.D.F. (BOCyL 9-2-10)
		\vspace*{2cm}
		Da Carmen Fernández Juncal\\
		Vicerrectora de Docencia\\
		Universidad de Salamanca
	}
}
\end{center}
\newpage

\textbf{Objetivos del seminario}
\begin{itemize}
\item Dar una introducción a la plataforma de desarrollo de videojuegos Unity3D.
\item Modelado bidimensional.
\item Programación de la lógica interna de un juego.
\item Desarrollo para diferentes dispositivos, integración de la plataforma con otros módulos, etcétera.
\end{itemize}

\vspace*{0.5cm}

\textbf{Contenido específico}
\begin{enumerate}
\item Instalación y configuración del entorno de desarrollo.
\item Conocimiento básico de los fundamentos de Unity3D.
\item Utilización y manipulación de elementos de un juego (escenario, personajes...).
\item Programación en Unity3D utilizando las herramientas integradas en el sistema.
\item Desarrollo para diferentes plataformas (ordenadores, móviles, táblets, consolas) e integración con componentes externos (librerías, frameworks...).
\end{enumerate}

\vspace*{0.7cm}

\begin{center}
El director del Centro de Formación Permanente\\
\vspace*{3cm}
Prof. Juan Luis García Alonso\\
Universidad de Salamanca
\end{center}
\end{document}